\section{Hétfő és Kedd}
\sordisz
\subsection{Reggeli: Zabkása mandulával és almával}
\begin{itemize}
    \item 50 g zabpehely
    \item 200 ml mandulatej
    \item 1 db alma
    \item 1 evőkanál mandula
    \item Fahéj
\end{itemize}
\textbf{Kalória:} 300 kcal \\
\textbf{Rost:} 8 g \\
\textbf{Elkészítés:} A zabpelyhet főzd fel a mandulatejben, add hozzá a reszelt almát és a fahéjat, majd szórd meg mandulával.

\sordisz
\subsection{Ebéd: Sárgarépa-krémleves pirított napraforgómaggal}
\begin{itemize}
    \item 500 g sárgarépa
    \item 1 fej vöröshagyma
    \item 2 gerezd fokhagyma
    \item 500 ml zöldségalaplé
    \item 2 evőkanál napraforgómag
    \item 1 evőkanál olívaolaj
\end{itemize}
\textbf{Kalória:} 320 kcal \\
\textbf{Rost:} 9 g \\
\textbf{Elkészítés:} A hagymát és fokhagymát pirítsd meg olívaolajon, add hozzá a felkockázott sárgarépát, majd főzd meg alaplében. Turmixold le, és szórd meg pirított napraforgómaggal.

\sordisz
\subsection{Vacsora: Sült cékla kecskesajttal és dióval}
\begin{itemize}
    \item 2 db cékla
    \item 50 g kecskesajt
    \item 1 evőkanál dió
    \item Rukkola
    \item 1 evőkanál balzsamecet
\end{itemize}
\textbf{Kalória:} 280 kcal \\
\textbf{Rost:} 7 g \\
\textbf{Elkészítés:} A céklát süsd meg 180°C-on 40 percig, majd szeleteld fel. Szórd meg kecskesajttal, dióval és rukkolával, majd locsold meg balzsamecettel.

\newpage

\section{Szerda és Csütörtök}
\sordisz
\subsection{Reggeli: Teljes kiőrlésű pirítós hummusszal és paprikával}
\begin{itemize}
    \item 1 szelet teljes kiőrlésű kenyér
    \item 2 evőkanál hummusz
    \item 1 db paprika
\end{itemize}
\textbf{Kalória:} 250 kcal \\
\textbf{Rost:} 6 g \\
\textbf{Elkészítés:} Kend meg a pirítóst hummusszal, és tálald felvágott paprikával.

\sordisz
\subsection{Ebéd: Zöldséges bulgur pirított tofuval}
\begin{itemize}
    \item 100 g bulgur
    \item 150 g tofu
    \item 1 db sárgarépa
    \item 1 db cukkini
    \item 1 evőkanál szójaszósz
    \item 1 evőkanál olívaolaj
\end{itemize}
\textbf{Kalória:} 380 kcal \\
\textbf{Rost:} 10 g \\
\textbf{Elkészítés:} A bulgurt főzd meg, a tofut pirítsd meg olívaolajon, majd add hozzá a feldarabolt zöldségeket és szójaszószt.

\sordisz
\subsection{Vacsora: Spenótos lencsesaláta fetával}
\begin{itemize}
    \item 200 g főtt lencse
    \item 50 g feta
    \item 100 g friss spenót
    \item 1 evőkanál tökmag
    \item Citromlé
\end{itemize}
\textbf{Kalória:} 320 kcal \\
\textbf{Rost:} 12 g \\
\textbf{Elkészítés:} A hozzávalókat keverd össze, és locsold meg citromlével.

\newpage

\section{Péntek}
\sordisz
\subsection{Reggeli: Kókuszos chiapuding bogyós gyümölcsökkel}
\begin{itemize}
    \item 3 evőkanál chia mag
    \item 150 ml kókusztej
    \item Friss bogyós gyümölcsök
    \item 1 evőkanál kókuszreszelék
\end{itemize}
\textbf{Kalória:} 260 kcal \\
\textbf{Rost:} 10 g \\
\textbf{Elkészítés:} A chia magot keverd össze a kókusztejjel, hagyd állni, majd tálald gyümölcsökkel és kókuszreszelékkel.

\sordisz
\subsection{Ebéd: Paradicsomos árpagyöngy ragu csicseriborsóval}
\begin{itemize}
    \item 100 g árpagyöngy
    \item 200 g főtt csicseriborsó
    \item 1 doboz paradicsompüré
    \item 1 fej vöröshagyma
    \item 1 evőkanál olívaolaj
\end{itemize}
\textbf{Kalória:} 370 kcal \\
\textbf{Rost:} 13 g \\
\textbf{Elkészítés:} A hagymát pirítsd meg, add hozzá a paradicsompürét és a csicseriborsót, majd forrald össze. Keverd hozzá a megfőzött árpagyöngyöt.

\sordisz
\subsection{Vacsora: Párolt brokkoli dióval és fokhagymás joghurttal}
\begin{itemize}
    \item 300 g brokkoli
    \item 1 evőkanál dió
    \item 100 g görög joghurt
    \item 1 gerezd fokhagyma
\end{itemize}
\textbf{Kalória:} 280 kcal \\
\textbf{Rost:} 9 g \\
\textbf{Elkészítés:} A brokkolit párold meg, szórd meg dióval, és tálald fokhagymás joghurttal.

\newpage
\section{Szombat}
\sordisz
\subsection{Reggeli: Tükörtojás teljes kiőrlésű pirítóssal és avokádóval}
\begin{itemize}
    \item 2 db tojás
    \item 1 szelet teljes kiőrlésű kenyér
    \item ½ db avokádó
    \item 1 teáskanál vaj
    \item Só, bors
\end{itemize}
\textbf{Kalória:} 350 kcal \\
\textbf{Rost:} 6 g \\
\textbf{Elkészítés:} A kenyeret pirítsd meg, a tojásokat süsd meg vajon, és tálald avokádóval.

\sordisz
\subsection{Ebéd: Grillezett csirkemell zöldségkörettel}
\begin{itemize}
    \item 150 g csirkemell
    \item 1 db cukkini
    \item 1 db sárgarépa
    \item 1 db paprika
    \item 1 evőkanál olívaolaj
    \item Só, bors, kakukkfű
\end{itemize}
\textbf{Kalória:} 420 kcal \\
\textbf{Rost:} 8 g \\
\textbf{Elkészítés:} A csirkét fűszerezd be, majd grillezd meg. A zöldségeket szeleteld fel, locsold meg olívaolajjal, és süsd meg grillen vagy serpenyőben.


\sordisz
\subsection{Vacsora: Tonhalsaláta olívával és főtt tojással}
\begin{itemize}
    \item 1 konzerv tonhal (sós lében)
    \item 50 g rukkola
    \item 1 db paradicsom
    \item ½ db uborka
    \item 1 evőkanál olívabogyó
    \item 1 db főtt tojás
    \item 1 evőkanál citromlé
\end{itemize}
\textbf{Kalória:} 340 kcal \\
\textbf{Rost:} 5 g \\
\textbf{Elkészítés:} A hozzávalókat keverd össze, és locsold meg citromlével.

\newpage

\section{Vasárnap}
\sordisz
\subsection{Reggeli: Házi túrós palacsinta teljes kiőrlésű lisztből}
\begin{itemize}
    \item 50 g teljes kiőrlésű liszt
    \item 1 db tojás
    \item 100 ml tej
    \item 100 g túró
    \item 1 evőkanál méz
    \item 1 teáskanál vanília kivonat
\end{itemize}
\textbf{Kalória:} 380 kcal \\
\textbf{Rost:} 4 g \\
\textbf{Elkészítés:} Keverd össze a palacsintatésztát, süsd ki, és töltsd meg túróval és mézzel.

\sordisz
\subsection{Ebéd: Sertéspörkölt teljes kiőrlésű nokedlivel}
\begin{itemize}
    \item 200 g sertéslapocka
    \item 1 fej vöröshagyma
    \item 1 gerezd fokhagyma
    \item 1 evőkanál pirospaprika
    \item 1 db paradicsom
    \item 1 evőkanál olaj
    \item Só, bors
    \item 100 g teljes kiőrlésű liszt (nokedlihez)
    \item 1 db tojás (nokedlihez)
    \item 50 ml víz (nokedlihez)
\end{itemize}
\textbf{Kalória:} 540 kcal \\
\textbf{Rost:} 7 g \\
\textbf{Elkészítés:} A hagymát pirítsd meg, add hozzá a húst és a fűszereket, majd főzd puhára. A nokedlihez keverd össze a lisztet, tojást és vizet, majd szaggasd forrásban lévő vízbe.

\sordisz
\subsection{Vacsora: Sült padlizsán feta sajttal és dióval}
\begin{itemize}
    \item 1 db padlizsán
    \item 50 g feta sajt
    \item 1 evőkanál dió
    \item 1 evőkanál olívaolaj
    \item Só, bors, bazsalikom
\end{itemize}
\textbf{Kalória:} 320 kcal \\
\textbf{Rost:} 6 g \\
\textbf{Elkészítés:} A padlizsánt szeleteld fel, süsd meg olívaolajon, majd tálald fetával és dióval.

\newpage

\section{Hozzávalók összesítése}
\sordisz
\begin{multicols}{2}
\subsection{Zöldségek és gyümölcsök}
\begin{itemize}
    \item 2 db avokádó
    \item 1 db cukkini
    \item 3 db sárgarépa
    \item 3 db paprika
    \item 2 db paradicsom
    \item 1 db uborka
    \item 1 fej vöröshagyma
    \item 1 gerezd fokhagyma
    \item 50 g rukkola
    \item 1 db padlizsán
    \item 1 db citrom
\end{itemize}

\subsection{Hús és hal}
\begin{itemize}
    \item 300 g csirkemell
    \item 200 g sertéslapocka
    \item 1 konzerv tonhal (sós lében)
\end{itemize}

\subsection{Tejtermékek és tojás}
\begin{itemize}
    \item 8 db tojás
    \item 100 ml tej
    \item 100 g túró
    \item 50 g feta sajt
\end{itemize}

\subsection{Gabona és lisztek}
\begin{itemize}
    \item 3 szelet teljes kiőrlésű kenyér
    \item 150 g teljes kiőrlésű liszt
\end{itemize}

\subsection{Olajok, fűszerek és egyéb}
\begin{itemize}
    \item 2 evőkanál olívaolaj
    \item 1 evőkanál vaj
    \item 1 evőkanál méz
    \item 1 teáskanál vanília kivonat
    \item 1 evőkanál pirospaprika
    \item Só, bors, kakukkfű, bazsalikom
\end{itemize}

\subsection{Magvak és olívabogyó}
\begin{itemize}
    \item 1 evőkanál dió
    \item 1 evőkanál olívabogyó
\end{itemize}
\end{multicols}