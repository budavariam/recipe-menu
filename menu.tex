\documentclass[a4paper,12pt]{article}
\usepackage[utf8]{inputenc}
\usepackage[margin=1in]{geometry}
\usepackage{titlesec}
\usepackage{array}
\usepackage{lmodern}
\usepackage{hyperref}
\usepackage{fancyhdr}
\usepackage{tikz}
\usepackage{multicol}

% Fejléc és lábléc beállítás
\pagestyle{fancy}
\fancyhf{}
\fancyhead[L]{\textbf{Otthoni Rostgazdag Menü}}
\fancyhead[R]{\today}
\fancyfoot[L]{Budavári Attila Mátyás}
\fancyfoot[R]{\thepage}

% Sordísz beállítása
\newcommand{\sordisz}{%
  \vspace{1em}
  \begin{center}
    \begin{tikzpicture}
      \draw[thick] (0,0) -- (0.3,0);
      \draw[thick] (0.7,0) -- (1.3,0);
      \draw[thick] (1.7,0) -- (2.3,0);
    \end{tikzpicture}
  \end{center}
  \vspace{1em}
}

% Dokumentum kezdete
\title{\Huge Otthoni Rostgazdag Menü}
\author{}
\date{}

\begin{document}

\maketitle
\tableofcontents
\newpage

\section{Hétfő és Kedd}
\sordisz
\subsection{Reggeli: Chia puding mandulatejjel}
\begin{itemize}
    \item 3 evőkanál chia mag
    \item 150 ml mandulatej
    \item Friss bogyós gyümölcsök (pl. áfonya, málna)
    \item 1 evőkanál mandula vagy dió
\end{itemize}
\textbf{Kalória:} 250 kcal \\
\textbf{Rost:} 10 g \\
\textbf{Elkészítés:} Áztasd be a chia magot mandulatejbe, hagyd állni 15-20 percig, majd keverd össze gyümölcsökkel és dióval.

\sordisz
\subsection{Ebéd: Sütőtökleves pirított tökmaggal}
\begin{itemize}
    \item 500 g sütőtök
    \item 1 fej vöröshagyma
    \item 2 gerezd fokhagyma
    \item 500 ml zöldségalaplé
    \item 2 evőkanál tökmag
    \item Teljes kiőrlésű kenyér
\end{itemize}
\textbf{Kalória:} 300 kcal \\
\textbf{Rost:} 8 g \\
\textbf{Elkészítés:} A sütőtököt és hagymát pirítsd meg, majd főzd meg alaplében, és turmixold össze. Szórd meg pirított tökmaggal.

\sordisz
\subsection{Vacsora: Lencsesaláta sült zöldségekkel}
\begin{itemize}
    \item 200 g főtt lencse
    \item 1 db répa
    \item 1 db cékla
    \item 1 db paprika
    \item 1 evőkanál dió
\end{itemize}
\textbf{Kalória:} 350 kcal \\
\textbf{Rost:} 12 g \\
\textbf{Elkészítés:} A zöldségeket süsd meg, majd keverd össze a lencsével és dióval.

\newpage

\section*{Hozzávalók összesítése}
\sordisz
\begin{multicols}{2}
\begin{itemize}
    \item Chia mag (250 g)
    \item Mandulatej (2-3 liter)
    \item Friss bogyós gyümölcsök
    \item Dió (200 g)
    \item Görög joghurt (1 kg)
    \item Teljes kiőrlésű kenyér
    \item Tojás (12 db)
    \item Avokádó (3-4 db)
    \item Sütőtök
    \item Barna rizs
    \item Brokkoli
    \item Fekete bab
    \item Quinoa
    \item Cékla
    \item Répa
    \item Paprika
\end{itemize}
\end{multicols}

\end{document}
