\documentclass[a4paper,12pt]{article}
\usepackage[utf8]{inputenc}
\usepackage[margin=1in]{geometry}
\usepackage{titlesec}
\usepackage{array}
\usepackage{lmodern}
\usepackage{hyperref}
\usepackage{multicol}

\titleformat{\section}{\large\bfseries}{\thesection.}{1em}{}
\titleformat{\subsection}{\normalsize\bfseries}{\thesubsection.}{0.5em}{}

\title{\Huge Heti Zabmentes Rostban Gazdag Menü}
\author{}
\date{}

\begin{document}

\maketitle
\tableofcontents
\newpage

\section{Hétfő és Kedd}
\subsection{Reggeli: Chia puding mandulatejjel}
\begin{itemize}
    \item 3 evőkanál chia mag
    \item 150 ml mandulatej
    \item Friss bogyós gyümölcsök (pl. áfonya, málna)
    \item 1 evőkanál mandula vagy dió
\end{itemize}
\textbf{Elkészítés:} Áztasd be a chia magot mandulatejbe, hagyd állni 15-20 percig, majd keverd össze gyümölcsökkel és dióval.

\subsection{Ebéd: Sütőtökleves pirított tökmaggal}
\begin{itemize}
    \item 500 g sütőtök
    \item 1 fej vöröshagyma
    \item 2 gerezd fokhagyma
    \item 500 ml zöldségalaplé
    \item 2 evőkanál tökmag
    \item Teljes kiőrlésű kenyér
\end{itemize}
\textbf{Elkészítés:} A sütőtököt és hagymát pirítsd meg, majd főzd meg alaplében, és turmixold össze. Szórd meg pirított tökmaggal.

\subsection{Vacsora: Lencsesaláta sült zöldségekkel}
\begin{itemize}
    \item 200 g főtt lencse
    \item 1 db répa
    \item 1 db cékla
    \item 1 db paprika
    \item 1 evőkanál dió
\end{itemize}
\textbf{Elkészítés:} A zöldségeket süsd meg, majd keverd össze a lencsével és dióval.

\newpage

\section{Szerda és Csütörtök}
\subsection{Reggeli: Görög joghurt bogyós gyümölcsökkel és dióval}
\begin{itemize}
    \item 200 g görög joghurt
    \item Friss bogyós gyümölcsök
    \item 1 marék dió
\end{itemize}
\textbf{Elkészítés:} A hozzávalókat keverd össze egy tálban.

\subsection{Ebéd: Töltött paprika bulgurral és zöldségekkel}
\begin{itemize}
    \item 2 db kaliforniai paprika
    \item 100 g bulgur
    \item 1 db cukkini
    \item 1 db paradicsom
    \item Paradicsomszósz
\end{itemize}
\textbf{Elkészítés:} Töltsd meg a paprikákat bulgurral és zöldségekkel, majd süsd meg.

\subsection{Vacsora: Avokádós pirítós tojással}
\begin{itemize}
    \item 1 szelet teljes kiőrlésű kenyér
    \item 1 db avokádó
    \item 1 db tojás
\end{itemize}
\textbf{Elkészítés:} A pirítósra kend az avokádót, és tedd rá a megsütött tojást.

\newpage

\section{Péntek}
\subsection{Reggeli: Smoothie spenóttal és banánnal}
\begin{itemize}
    \item 1 marék spenót
    \item 1 db banán
    \item 200 ml mandulatej
    \item 1 evőkanál mandula
\end{itemize}
\textbf{Elkészítés:} A hozzávalókat turmixold össze.

\subsection{Ebéd: Grillezett lazac barna rizzsel és brokkolival}
\begin{itemize}
    \item 150 g lazac
    \item 100 g barna rizs
    \item 200 g brokkoli
    \item Citrom
\end{itemize}
\textbf{Elkészítés:} A lazacot grillezd meg, és tálald a rizzsel és párolt brokkolival.

\subsection{Vacsora: Quinoa saláta sült édesburgonyával}
\begin{itemize}
    \item 100 g quinoa
    \item 1 db édesburgonya
    \item 100 g fekete bab
    \item Kukorica
\end{itemize}
\textbf{Elkészítés:} A quinoát főzd meg, és keverd össze a sült édesburgonyával, babbal és kukoricával.

\newpage

\section{Szombat és Vasárnap}
\subsection{Reggeli: Tojásos zöldséges omlett}
\begin{itemize}
    \item 3 db tojás
    \item 1 db paprika
    \item 1 marék spenót
    \item 100 g gomba
\end{itemize}
\textbf{Elkészítés:} Készíts omlettet a zöldségekkel.

\subsection{Ebéd: Padlizsánkrém teljes kiőrlésű kenyérrel}
\begin{itemize}
    \item 1 db padlizsán
    \item 1 gerezd fokhagyma
    \item 1 evőkanál olívaolaj
    \item Teljes kiőrlésű kenyér
\end{itemize}
\textbf{Elkészítés:} A padlizsánt süsd meg, majd turmixold össze fokhagymával és olívaolajjal.

\subsection{Vacsora: Céklás lencsesaláta}
\begin{itemize}
    \item 200 g főtt lencse
    \item 1 db főtt cékla
    \item 1 evőkanál dió
    \item Friss petrezselyem
\end{itemize}
\textbf{Elkészítés:} A lencsét és céklát keverd össze dióval és petrezselyemmel.

\newpage

\section*{Hozzávalók összesítése}
\begin{multicols}{2}
\begin{itemize}
    \item Chia mag (250 g)
    \item Mandulatej (2-3 liter)
    \item Friss bogyós gyümölcsök
    \item Dió (200 g)
    \item Görög joghurt (1 kg)
    \item Teljes kiőrlésű kenyér
    \item Tojás (12 db)
    \item Avokádó (3-4 db)
    \item Sütőtök
    \item Barna rizs
    \item Brokkoli
    \item Fekete bab
    \item Quinoa
    \item ...
\end{itemize}
\end{multicols}

\end{document}
